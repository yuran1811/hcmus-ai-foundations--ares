\subsection{Uniform Cost Search}
\noindent For any search problem, Uniform Cost Search (UCS) is the better algorithm than the previous ones. The search algorithm explores in branches with more or less same cost. This consist of a priority queue where the path from the root to the node is the stored element and the depth to a particular node acts as the priority. UCS assumes all the costs to be non negative. While the DFS algorithm gives maximum priority to maximum depth, this gives maximum priority to the minimum cumulative cost.

\subsubsection{Pseudocode}
\begin{algorithm}[H]
	\caption{Uniform Cost Search (\textit{start, goal})}
	\label{alg:ucs}
	\begin{algorithmic}[1]
	\State priority queue \(\gets\) [(start, cost = 0)]
	\While {priority queue is not empty}
		\State (node, cost) \(\gets\) dequeue(priority queue)
		\If {node = goal}
			\State return path
		\EndIf
		\ForAll {neighbor in valid moves}
			\State new cost \(\gets\) cost + move cost
			\If {neighbor not visited or new cost \(<\) previous cost}
				\State mark neighbor as visited
				\State enqueue(priority queue, (neighbor, new cost))
			\EndIf
		\EndFor
	\EndWhile
	\State return failure
	\end{algorithmic}
\end{algorithm}

\subsubsection{Implementation}

\subsubsection{Time and Space Complexity}

