\subsection{Pruning Techniques}
\noindent Pruning is a terminology used in machine learning and artificial intelligence that are used to reduce the size of the decision trees by removing the selected sections of the tree that provide undesirable results. We have implemented two techniques till now in the project and are described below:
\subsubsection{Move appension}
\noindent One of the primary problem in search problems is that the computation time becomes unimaginably high when the search space is big. In such cases, if we have priori knowledge about the systems, we can limit in the beginning of the search problem all the cases where we have impossible actions. For instance, the Figure 3 depicts a case where the only acceptable action is to move Up. The possible actions for any algorithm can be moving in all the directions which can be reduced to one by Move Appension where we restrict all the impossible actions.
\subsubsection{Hashing}
\noindent Hashing is a well known pruning method used to tune the algorithm to perform better. It follows the logic that decision which leads to the states that are already visited are considered as suboptimal. So all the states are stored in the hash table and at each point, a comparison is made between the current state and the stored state. If there is a match, the same action corresponding to the one in the hashing table is avoided
\subsubsection{Tunnel Macros}
\noindent Tunnel Macros is a intelligent pruning technique which is employed to reduce the state space exploration. Very often, the Sokoban game consist of the tunnels where the stones are required to be pushed. In such cases, the action is not going to change till the tunnel end is reached. Therefore, if we can cut down the actions in the tunnels by identifying them, the time required to solve the level decreases exponentially.
